
%-----Chapter 6: FlawsOutlook-------%
\chapter{Challenges and Outlook}

Several problems which can occur in a real Nao soccer match are covered by our design of the model such as measurement drops, measurement fusion or the fact that inputs of the opposing team are unknown. However there are still practical challenges which we do not encounter in our work. Maybe the most important challenge is the provision of measurements and their related covariances. Up to this point we simply assumed that there is an algorithm which computes the positions and directions of objects on the playing field. We didn't care about the fact whether this can be done in a reliable way or within a reasonable space of time. This task was beyond the scope of this work and certainly could be the subject of another whole group work for the people in the ETHZ RoboCup Team which are occupied with perception.
\parskip 0pt

Other problems arise from the restrictions of our chosen model, for example the assumption that there are only three scenarios under which measurements are gathered. In reality we are probably faced with much more possibilities, that do not only depend on whether a robot sees a landmark or not but also on the actual sight distance to an object or the head angle of a robot. Furthermore the measurement noise may be coloured and not white, so our model would be inaccurate in describing the uncertainty in the given system. Those kind of issues can be summarized as ''lack of detail'', the approximation to reality is not close enough. This can be solved by a better and deeper analysis of the system.

Last but not least there are still several things, that were only partly or not at all considered in our work and which have huge potential for the further localization task. One main aspect here is certainly the practical application of the simulation on the Nao Platform. Not only that it is of course the main goal to have a working estimation and prediction for a real Nao soccer match but also that further improvements of the estimation algorithms will only be possible if they are tested in real environment with real constraints. The modularity of our simulation allows improvements in specific areas concerning estimation. For example the tracking of the ball may be improved too by replacing the extended Kalman filter by a so called particle filter, so switching from deterministic to probabilistic methods. It is also imaginable to have hybrid forms like it is partly done for the extended Kalman filter of the robots (see Ch. 4.1). As a third point one could mention the vast improvements which are possible in estimation if the robots are not acting randomly but if they have a strategy that governs their movements. In terms of localization that would mean that blue robots actively seek getting good measurements and that for example losing the current position of the ball for long periods is not possible anymore. More extensions of the current basic configuration such as better input approximation of enemy robots, the prediction of enemy movements or better handling of discrete events are also thinkable and can have positive effects on the overall performance.



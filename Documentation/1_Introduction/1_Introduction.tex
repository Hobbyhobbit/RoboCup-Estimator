
%-----Chapter 1: Introduction-------%
\chapter{Introduction}
RoboCup ("Robot Soccer World Cup") is an international ongoing robotics competition founded in 1997. The official goal of the project: "By mid-21st century, a team of fully autonomous humanoid robot soccer players shall win the soccer game, complying with the official rule of the FIFA, against the winner of the most recent World Cup." \cite{wwwRoboCup}
\parskip 0pt

There are a lot of topics of automation and control contained in this project. One is the topic of estimation, which is a part of our group work. The goal of our work is to build a cooperative estimation and prediction of player and ball movement. A detailed description and further informations about the goal of our group work can be found in the appendix \ref{PrjDescription}.

But why do we actually need estimation of movements, is it not enough to simply take the measurements we get and use them as estimates? The answer to that question is no for several reasons. Since the measurements are gathered by the built-in cameras of the robots, they are subject to measurement noise. This noise is certainly much bigger than the noise of a global eye which would see the whole playing field. So one reason to use estimation is the need to get rid of this measurement noise. An other reason concerns the availability of measurements. It is easily possible that objects on the field will be measured more than one time or not at all within one timestep. For the former case it is important that we have a meaningful measurement fusion such that we can take advantage of the multiplicity of measurements. For the later case we would like to keep track of objects even if measurements are not available. Therefore we need some kind of prediction. For the prediction task on the other side we have the problem of asymmetric information distribution. That means that input parameters, like the velocity of objects, are perfectly known for own team players but not for the opposite team and the ball. A filter task gets even harder with these constraints.

Our approach is now to solve all of these problems with a single estimation and prediction module. One can think of a block model with an input and an output. The input is the data coming from all four robot cameras which we have at our disposal. This data is then processed by the estimation unit; we estimate the position of our own robots as well as the position of other robots and the ball. This step consists of fusing and filtering all incoming measurements. The output will finally be a global map of the current situation on the playing field, so we try to simulate a global eye by using the equipment and possibilities our robots have. The following chapters of this documentation will explain step by step how we tried to solve this task.

